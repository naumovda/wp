\chapter{ПЕРЕЧЕНЬ ПЛАНИРУЕМЫХ РЕЗУЛЬТАТОВ ОБУЧЕНИЯ ПО ДИСЦИПЛИНЕ, СООТНЕСЕННЫХ С ПЛАНИРУЕМЫМИ РЕЗУЛЬТАТАМИ ОСВОЕНИЯ ОБРАЗОВАТЕЛЬНОЙ ПРОГРАММЫ} \label{chapt1}
Рабочая программа по дисциплине \DisciplineName~ является составной частью основной профессиональной образовательной программы по специальности \SpecialityCode~\SpecialityName, разработанной в соответствии с Федеральным государственным образовательным стандартом высшего образования по специальности \SpecialityCode~\SpecialityName~ (\QualificaionLevel), утвержденным \DocumentApprove.

Цель дисциплины \DisciplineName ~-– повышение исходного уровня владения иностранным языком, достигнутого на предыдущей ступени образования, и овладения студентами необходимым и достаточным уровнем коммуникативной компетенции для решения социально-коммуникативных задач в различных областях бытовой, культурной, профессиональной и научной деятельности при общении с зарубежными партнерами, при подготовке научных работ, а также для дальнейшего самообразования.

Основные задачи освоения учебной дисциплины:
\begin{enumerate}
\item Формирование социокультурной компетенции и поведенческих стереотипов, необходимых для успешной адаптации выпускников на рынке труда; 
\item Развитие у студентов умения самостоятельно приобретать знания для осуществления бытовой и профессиональной коммуникации на иностранном языке – повышение уровня учебной автономии, способности к самообразованию, к работе с мультимедийными программами, электронными словарями, иноязычными ресурсами сети Интернет; 
\item Развитие когнитивных и исследовательских умений, расширение кругозора и повышение информационной культуры студентов; 
\item Формирование представления об основах межкультурной коммуникации, воспитание толерантности и уважения к духовным ценностям разных стран и народов; 
\item Расширение словарного запаса и формирование терминологического аппарата на иностранном языке в пределах профессиональной сферы.
\end{enumerate}

В таблице \ref{tblCompetentions} приведены коды компетенций, содержание компетенций и перечень планируемых результатов обучения по дисциплине.

\begin{table}[pht]%
	\caption{Компетенции дисциплины}
	\label{tblCompetentions}	
	\begin{tabularx}{\textwidth}{p{.1\textwidth}p{.3\textwidth}X}
        \toprule
    	Код & Содержание \par компетенций & Перечень планируемых результатов обучения по дисциплине\\
        \midrule 
  		УК-4 & Способен осуществлять деловую коммуникацию в устной и письменной формах на государственном языке Российской Федерации и иностранном(ых) языке(ах) & 
  		УК–4.1 Знать способы выражения определенных коммуникативных намерений, речевые тактики профессиональной коммуникации, психологические аспекты речевой коммуникации; \par
  		УК–4.2 Знать грамматические, лексические, стилистические особенности иностранного языка в социокультурной и профессиональной сфере; \par 
  		УК-4.3 Уметь обмениваться информацией профессионального и делового характера на русском и иностранном языках в устной и письменной форме; \par
  		УК-4.4 Уметь соотносить языковые средства с конкретными сферами, ситуациями, условиями и задачами общения;\par 
  		УК-4.5 Владеть деловым речевым этикетом, специфичным сфере деятельности; \par
		УК-4.6 Владеть широким кругозором в научной и специальной сферах; \par
		УК-4.7 Владеть навыками и умениями точного понимания содержания текста на иностранном языке на основе его информационной переработки.\\
        \midrule		
  		ОПК-3 & Способен самостоятельно представлять научные результаты, составлять научные документы и отчеты & 
  		ОПК-3.1 Владеть навыками реализации  коммуникативных намерений в профессиональных и научных целях в устной и письменной форме; \par
		ОПК-3.2 Владеть иностранным языком на уровне, достаточном для осуществления творческой деятельности; \par
		ОПК-3.3 Владеть навыками составления документации для делового и научного общения; \\
        \bottomrule
	\end{tabularx}
\end{table}


\chapter{МЕСТО ДИСЦИПЛИНЫ В СТРУКТУРЕ ОБРАЗОВАТЕЛЬНОЙ ПРОГРАММЫ}\label{chapt2}

Дисциплина \DisciplineName~ является \DisciplineNeed~дисциплиной, относится к блоку №\BlockCode~ основной профессиональной образовательной программы (ОПОП) по направлению подготовки \SpecialityCode~\SpecialityName~\OrganizationShortName.

Дисциплина изучается по формам обучения: \FormOfStudy, в семестрах: \SemestryList.

До начала изучения учебной дисциплины обучающиеся должны знать, уметь и владеть следующими навыками:
\begin{enumerate}
\item Знать:
	\begin{itemize}
	\item основные способы словообразования;
	\item грамматические основы изучаемого иностранного языка в рамках средней школы;
	\item лексику в объеме и в рамках тематик, предусмотренных программой средней школы;
	\item интонацию различных коммуникативных типов предложений;
	\item основные нормы речевого этикета (реплики-клише, наиболее распространенную оценочную лексику), принятые в стране изучаемого языка;
	\item роль владения иностранными языками в современном мире, особенности образа жизни, быта, культуры стран изучаемого языка, сходство и различия в традициях своей страны и стран изучаемого языка.	
	\end{itemize}
\item Уметь (в области говорения):
	\begin{itemize}
	\item начинать, вести/поддерживать и заканчивать беседу в стандартных ситуациях общения, соблюдая нормы речевого этикета;
	\item расспрашивать собеседника и отвечать на его вопросы, высказывая свое мнение, просьбу, отвечать на предложение собеседника согласием/отказом, опираясь на изученную тематику и усвоенный лексико-грамматический материал;
	\item делать краткие сообщения, описывать события/явления (в рамках пройденных тем), передавать основное содержание, основную мысль прочитанного или услышанного, выражать свое отношение к прочитанному/услышанному, давать краткую характеристику персонажей;
	\item использовать перифраз, синонимичные средства в процессе устного общения;
	\end{itemize}
\item Уметь (в области аудирования):
	\begin{itemize}
	\item понимать основное содержание кратких, несложных аутентичных прагматических текстов и выделять для себя значимую информацию;
	\item понимать основное содержание несложных аутентичных текстов, относящихся к разным коммуникативным типам речи (сообщение/рассказ), уметь определить тему текста, выделить главные факты в тексте, опуская второстепенные;
	\item использовать переспрос, просьбу повторить;
	\end{itemize}
\item Уметь (в области в области чтения):
	\begin{itemize}
	\item ориентироваться в иноязычном тексте: прогнозировать его содержание по заголовку;
	\item  читать аутентичные тексты разных жанров преимущественно с пониманием основного содержания;
	\item читать несложные аутентичные тексты разных жанров с полным и точным пониманием, используя различные приемы смысловой переработки текста (языковую догадку, анализ, выборочный перевод), оценивать полученную информацию, выражать свое мнение;
	\item читать текст с выборочным пониманием нужной или интересующей информации;
	\end{itemize}
\item Уметь (в области письма):
	\begin{itemize}
	\item заполнять анкеты и формуляры;
	\item писать поздравления, личные письма с опорой на образец, употребляя формулы речевого этикета, принятые в странах изучаемого языка.
	\end{itemize}
\item Уметь использовать приобретенные знания и умения в практической деятельности и повседневной жизни для:
	\begin{itemize}
	\item социальной адаптации; достижения взаимопонимания в процессе устного и письменного общения с носителями иностранного языка, установления межличностных и межкультурных контактов в доступных пределах;
	\item создания целостной картины полиязычного, поликультурного мира, осознания места и роли родного и изучаемого иностранного языка в этом мире;
	\item приобщения к ценностям мировой культуры как через иноязычные источники информации, в том числе мультимедийные, так и через участие в школьных обменах, туристических поездках, молодежных форумах;
	\item ознакомления представителей других стран с культурой своего народа; осознания себя гражданином своей страны и мира..
	\end{itemize}
\item Владеть:
	\begin{itemize}
	\item слухо-произносительными и орфографическими навыками по изучаемому иностранному языку;
	\item навыками продуктивного использования основных грамматических форм и конструкций;
	\item лексическим запасом в бытовой, учебно-познавательной и социально-культурных сферах общения в рамках средней школы;
	\item всеми видами речевой деятельности на иностранном языке в объеме, предусмотренной программой средней школы.
	\end{itemize}
\end{enumerate}	

\chapter{ОБЪЕМ ДИСЦПИЛИНЫ}
\label{chapt3}

В разделе указан объем дисциплины в зачетных единицах с указанием количества академических часов, выделенных на контактную работу обучающихся с преподавателем (по видам занятий) и на самостоятельную работу обучающихся.

Общая трудоемкость дисциплины составляет в 8 зачетных единицы (ЗЕ).

Объем дисциплины в зачетных единицах с указанием количества академических часов, выделенных на контактную работу обучающихся с преподавателем (по видам занятий) и на самостоятельную работу обучающихся приведен в таблице \ref{tblHours}.

\begin{table}
  \caption{Трудоемкость дисциплины}
  \label{tblHours}  
  \begin{tabularx}{\textwidth}{p{.1\textwidth} X p{.1\textwidth}}
  \toprule  
  № & Вид учебной работы & Часы\\
  \midrule   
  1 & Общая трудоемкость дисцилины,\par в том числе: & 288\\
  \midrule   
  1.1 & контактная работа обучающихся  (всего),\par в том числе: & 112\\  
  1.1.1 & - лекции & ~--\\    
  1.1.2 & - лабораторные работы & ~--\\  
  1.1.3 & - практические занятия & 112\\    
  \midrule   
  1.2 & самостоятельная работа обучающихся (всего),\par в том числе: & 122\\    
  1.2.1 & - курсовой проект & ~--\\      
  1.2.2 & - курсовая работа & ~--\\        
  1.2.3 & - иные виды самостоятельной работы & 122\\          
  \midrule     
  1.3 & контроль (всего),\par в том числе: & ~54\\    
  1.3.1 & - подготовка к экзамену & ~54\\
  1.3.2 & - консультации & ~--\\  
  \bottomrule  
  \end{tabularx}  
\end{table}

Вид промежуточной аттестации обучающихся: зачет (3 семестр).

\chapter{СОДЕРЖАНИЕ ДИСЦИПЛИНЫ}
\label{chapt4}

В разделе приведено содержание дисциплины, структурированное по темам (разделам) с указанием отведенного на них количества академических часов и видов учебных занятий

\section{Содержание дисциплины, структурированное по темам}

В структурном отношении программа дисциплины представлена следующими темами:
\begin{enumerate}
\item \textit{Тема 1. The history of personal computers (История персональных компьютеров)}\par
Введение в дисциплину. Словообразование: Словопроизводство без изменения произношения и написания слова. Грамматика: Глагол. Общие сведения. Личные и неличные формы. Залог.  Действительный залог. Способы выражения действия в будущем времени в английском языке.
Выполнение практических заданий по всем видам речевой деятельности в рамках тематики модуля. 
\item \textit{Тема 2. Computers. Further development. (Компьютеры. Дальнейшее развитие).} \par	
Словообразование: Словопроизводство при помощи изменения места ударения. Грамматика: Страдательный залог. Выполнение практических заданий по всем видам речевой деятельности в рамках тематики модуля. 
\item \textit{Тема 3. Parts of computer systems. Hardware. (Части компьютерной системы. Аппаратное обеспечение)} \par	Словообразование: Словопроизводство при помощи чередования звуков.
Грамматика: Имя существительное. Общие сведения. Образование множественного числа имен существительных. Падеж имен существительных. Имена существительные в роли определения. Грамматическая омонимичность слов, оканчивающихся на –s. 
Выполнение практических заданий по всем видам речевой деятельности в рамках тематики модуля. Написание доклада.
\item \textit{Тема 4. Parts of computer systems. Software. (Части компьютерной системы. Программное обеспечение)}	
\par Словообразование: Словопроизводство. Префиксы с отрицательным значением. Грамматика: Местоимение. Общие сведения. Личные, притяжательные и возвратно-усилительные местоимения. Указательные местоимения. Выполнение практических заданий по всем видам речевой деятельности в рамках тематики модуля. 
\item \textit{Тема 5. The history of computer hardware. (История развития аппаратного обеспечения)} \par	
Словообразование: Словопроизводство. Префиксы с разными значениями.
Грамматика: Неопределенные местоимения. Выполнение практических заданий по всем видам речевой деятельности в рамках тематики модуля. Проведение дискуссии. Написание эссе об истории развития аппаратного обеспечения.
\item \textit{Тема 6. Central processing unit and microprocessor. (Центральный процессор и микропроцессор)}\par
Словообразование: Словопроизводство. Суффиксы. Грамматика: Имя прилагательное. Общие сведения. Степени сравнения прилагательных.
Выполнение практических заданий по всем видам речевой деятельности в рамках тематики модуля. Поиск дополнительной информации по тематике модуля с помощью информационных ресурсов.
\item \textit{Тема 7. Computer software. (Компьютерное ПО)} \par	Словообразование: Словопроизводство. Суффиксы.
Грамматика: Согласование времен и косвенная речь. Общие сведения. Правила согласования времен. Перевод прямой речи в косвенную речь.
Выполнение практических заданий по всем видам речевой деятельности в рамках тематики модуля. 
\item \textit{Тема 8. Types of personal computers. (Типы персональных компьютеров)} \par	Словообразование: Словопроизводство. Суффиксы. Грамматика: Модальные глаголы. Общие сведения. Наиболее употребительные модальные глаголы и их эквиваленты.
Выполнение практических заданий по всем видам речевой деятельности в рамках тематики модуля. Комментирование высказывания.
\item \textit{Тема 9. Computer applications. (Прикладные вычислительные системы)} \par 
Словообразование: Словопроизводство. Суффиксы существительных.
Грамматика: Числительное. Общие сведения. Образование количественных числительных. Образование порядковых числительных. Некоторые особенности употребления числительных в английском языке.
Выполнение практических заданий по всем видам речевой деятельности в рамках тематики модуля. 
\item \textit{Тема 10. Peripherals. (Периферийные устройства)} \par	Словообразование: Словопроизводство. Суффиксы прилагательных.
Грамматика: Предложение. Структура простого предложения.  Виды предложений. Типы придаточных предложений. Бессоюзные придаточные предложения. Выполнение практических заданий по всем видам речевой деятельности в рамках тематики модуля. 
\item \textit{Тема 11. Operating systems. (Операционные системы)} \par
Словообразование: Словопроизводство. Суффиксы глаголов. Грамматика: Инфинитив. Общие сведения. Инфинитив без частицы to. Формы инфинитива. Функции инфинитива. Выполнение практических заданий по всем видам речевой деятельности в рамках тематики модуля. Поиск информации в интернете о типах операционных систем.  Написание доклада по результатам исследования.
\item \textit{Тема 12. Computer storage. (Запоминающее устройство)}\par
Словообразование: Словопроизводство. Суффиксы наречий.
Грамматика: Сложное дополнение / Объектный инфинитивный оборот. Сложное подлежащее / Субъектный инфинитивный оборот. Инфинитивная конструкция с предлогом for.
Выполнение практических заданий по всем видам речевой деятельности в рамках тематики модуля. Практика перевода с русского языка на английский.
\item \textit{Тема 13. Application programs. (Прикладные программы)}\par
Словообразование: Словопроизводство. Суффиксы и префиксы (обобщение). Грамматика: Причастие. Общие сведения. Формы причастия. Функции Present Participle (Participle I). Функции Past Participle (Participle II). Выполнение практических заданий по всем видам речевой деятельности в рамках тематики модуля. Подготовка доклада на тему «Прикладные программы».
\item \textit{Тема 14. Multimedia (Мультимедиа)}\par
Словообразование: Словосложение. Грамматика: Объектный причастный оборот. Субъектный причастный оборот. Независимый причастный оборот. Грамматическая омонимичность слов, оканчивающихся на -ed.
Выполнение практических заданий по всем видам речевой деятельности в рамках тематики модуля. 
\item \textit{Тема 15. Software engineering (Программотехника)}\par
Словообразование: Словосложение. Фразовые глаголы.Грамматика: Герундий. Общие сведения. Формы герундия.  Функции герундия. Герундиальный оборот. Грамматическая омонимичность слов, оканчивающихся на -ing. Выполнение практических заданий по всем видам речевой деятельности в рамках тематики модуля. 
\item \textit{Тема 16. Computer networks (Компьютерные сети)}\par
Грамматика: Сослагательное наклонение. Типы условных предложений. Выполнение практических заданий по всем видам речевой деятельности в рамках тематики модуля. Подготовка доклада на тему «Компьютерные сети».
\item \textit{Тема 17. The internet (Интернет)}\par
Грамматика: Многофункциональные местоимения. Выполнение практических заданий по всем видам речевой деятельности в рамках тематики модуля. Практика перевода с русского языка на английский.
\item \textit{Тема 18. Websites (Веб-сайты)}\par
Грамматика: Вспомогательные глаголы. Выполнение практических заданий по всем видам речевой деятельности в рамках тематики модуля. Написание эссе: «Веб-сайты».
\item \textit{Тема 19. Communication systems (Системы коммуникации)}\par
Грамматика: Многофункциональные, многозначные слова. Выполнение практических заданий по всем видам речевой деятельности в рамках тематики модуля. Групповое исследование по тематике модуля.
\item \textit{Тема 20. The future of IT (Будущее информационных технологий)}\par
Грамматика: Многофункциональные, многозначные слова (продолжение). Выполнение практических заданий по всем видам речевой деятельности в рамках тематики модуля. Написание доклада «Будущее информационных технологий».
\end{enumerate}

\section{Тематический план дисциплины}
Тематический план дисциплины (таблица \ref{tbl:Plan}) включает следующие формы учебного процесса:
\begin{itemize}
\item лекции (ЛЕК); 
\item лабораторные работы (ЛАБ); 
\item практические занятия и семинары (ПЗ); 
\item самостоятельную работу (СР).
\end{itemize}

%\begin{table}[htp]
%  \centering
%  \caption{Sample} \vspace{3pt}
%  \begin{spreadtab}{{tabular}{|c|l|l|c|c|c|c|c|c|}}
%    \hline
%    @\multirow{2}{*}{Sl No.} &
%    @\multicolumn{2}{c}{\multirow{2}{*}{\textbf{something}}} &
%    @\multicolumn{4}{|c|}{\textbf{s}} & @\textbf{a~/~b} &
%    @\multirow{2}{*}{\textbf{Total}} \\
%    \cline{4-7}
%    & @\multicolumn{2}{|c|}{} & @\textbf{x} & @\textbf{y} & @\textbf{z}
%    & @\textbf{m} & @\textbf{/~c} & \\ 
%    \hline
%    @\multirow{2}{*}{1} & @1 \& & @x &2 &3 &2 & 1 & 3&
%    \STcopy{v}{d3+e3+f3+g3}
%    \\
%    \cline{3-9}
%    & @Safety & @\textbf{Sub Total} & 3 & 3 & 9 & 7 &3 & \\ \hline
%    @\multirow{2}{*}{2} & @2 \& & @x &6 &3 &5 & 1 & 3&
%    \\
%    \cline{3-9}
%    & @Safety & @\textbf{Sub Total} & 2 & 10 & 1 & 7 &3 & \\ \hline
%  \end{spreadtab}
%\end{table}

\begin{table}[pht]%
	\caption{Тематический план}%
	\label{tbl:Plan}
    %\setlength\extrarowheight{6pt} 
    %\setlength{\tymin}{1.9cm}
	\begin{tabulary}{\textwidth} {@{} >{\zz}C >{\zz}L >{\zz}C >{\zz}C >{\zz}C >{\zz}C >{\zz}C >{\zz}C @{}}
        \toprule
        \multirow{2}{*}{№} & \multirow{2}{*}{Тема} & \multirow{2}{*}{Всего, час.} & \multicolumn{4}{c}{Контактная работы} & СР, час.\\ 
		& & & Всего, час.& ЛЕК, час.& ЛАБ, час.& ПЗ, час.&  \\
        \midrule 
        1 & The history of personal computers   & 14 & 6 & 0 & 0 & 6 & 8 \\
        2 & Computers. Further development      & 14 & 6 & 0 & 0 & 6 & 8 \\
        3 & Parts of computer systems. Hardware & 14 & 6 & 0 & 0 & 6 & 8 \\        
        4 & Parts of computer systems. Software & 14 & 6 & 0 & 0 & 6 & 8 \\                
        5 & The history of computer hardware    & 14 & 6 & 0 & 0 & 6 & 8 \\                        
        6 & Central processing unit and microprocessor & 14 & 6 & 0 & 0 & 6 & 8 \\                                
        7 & Computer software & 14 & 6 & 0 & 0 & 6 & 8 \\                                        
        8 & Types of personal computers & 14 & 6 & 0 & 0 & 6 & 8 \\                                                
        9 & Computer applications & 14 & 6 & 0 & 0 & 6 & 8 \\                                                        
        10 & Peripherals & 14 & 6 & 0 & 0 & 6 & 8 \\                                                                
        11 & Operating systems & 14 & 6 & 0 & 0 & 6 & 8 \\                                                                        
        12 & Computer storage & 14 & 6 & 0 & 0 & 6 & 8 \\                                                                                
        13 & Application programs & 14 & 6 & 0 & 0 & 6 & 8 \\                                                                                        
        14 & Multimedia  & 14 & 6 & 0 & 0 & 6 & 8 \\                                                                                                
        15 & Software engineering & 14 & 6 & 0 & 0 & 6 & 8 \\                                                                                                        
        16 & Computer networks & 14 & 6 & 0 & 0 & 6 & 8 \\                                                                                                                
        17 & The internet & 14 & 6 & 0 & 0 & 6 & 8 \\                                                                                                                        
        18 & Websites & 14 & 6 & 0 & 0 & 6 & 8 \\                                                                                                                                
        19 & Communication systems & 14 & 6 & 0 & 0 & 6 & 8 \\                                                                                                                                        
        20 & The future of IT & 14 & 6 & 0 & 0 & 6 & 8 \\                                                                                                                                                
        \midrule%%% тонкий разделитель
        ~ & Всего: & 234 & 112 & 0 & 0 & 112 & 122 \\
        ~ & Всего с контролем: & 288 &   &   &   &   &   \\        
        \bottomrule %%% нижняя линейка
	\end{tabulary}%
	%\end{longtable}
\end{table}

\begin{center}
\begin{longtable}{p{0.05\textwidth} p{0.15\textwidth} p{0.1\textwidth} p{0.5\textwidth} p{0.05\textwidth}}
\caption{Перечень практических и самостоятельных работ}\\
\hline
\textbf{№} & \textbf{Тема} & \textbf{Вид} & \textbf{Содержание работы} & Часы \\
\hline
1 & 2 & 3 & 4 & 5\\ 
\hline
\endfirsthead
\multicolumn{4}{c}%
{\tablename\ \thetable\ -- \textit{Продолжение}} \\
\hline
1 & 2 & 3 & 4 & 5\\ 
\hline
\endhead
\hline \multicolumn{4}{r}{\textit{Продолжене на сл. стр.}} \\
\endfoot
\hline
\endlastfoot
1 & History of personal computers & СР, ПР & Практика перевода с английского языка на русский. Выполнение практических заданий по всем видам речевой деятельности в рамках тематики модуля. 
Составление резюме текста. Написание эссе. Лексико-грамматическое тестирование & 8\\
2 & Computers. Further development & СР, ПР & Практика перевода с английского языка на русский. Выполнение практических заданий по всем видам речевой деятельности в рамках тематики модуля. 
Лексико-грамматическое тестирование & 8 \\
3 & Parts of computer systems. Hardware & СР, ПР & Практика перевода с английского языка на русский. Выполнение практических заданий по всем видам речевой деятельности в рамках тематики модуля. Написание доклада. 
Лексико-грамматическое тестирование & 8 \\
4 & Parts of computer systems. Software & СР, ПР & Практика перевода с английского языка на русский. Выполнение практических заданий по всем видам речевой деятельности в рамках тематики модуля. Лексико-грамматическое тестирование & 8 \\
5 & The history of computer hardware & CР, ПР & Практика перевода с английского языка на русский. Выполнение практических заданий по всем видам речевой деятельности в рамках тематики модуля. 
Проведение дискуссии. Написание эссе об истории развития аппаратного обеспечения. 
Лексико-грамматическое тестирование & 8 \\
6 & Central processing unit and microprocessor & СР, ПР & Практика перевода с английского языка на русский. Выполнение практических заданий по всем видам речевой деятельности в рамках тематики модуля. 
Поиск дополнительной информации по тематике модуля с помощью информационных ресурсов. Лексико-грамматическое тестирование & 8 \\
7 & Computer software & СР, ПР & Практика перевода с английского языка на русский. Выполнение практических заданий по всем видам речевой деятельности в рамках тематики модуля. Описание технологических новинок и их применений. Лексико-грамматическое тестирование. & 8 \\
8 & Types of personal computers & СР, ПР & Практика перевода с английского языка на русский. Выполнение практических заданий по всем видам речевой деятельности в рамках тематики модуля. Комментирование высказывания. 
Лексико-грамматическое тестирование & 8 \\
9 & Computer applications & СР, ПР & Практика перевода с английского языка на русский. Выполнение практических заданий по всем видам речевой деятельности в рамках тематики модуля. Лексико-грамматическое тестирование & 8 \\
10 & Peripherals & СР, ПР & Практика перевода с английского языка на русский. Выполнение практических заданий по всем видам речевой деятельности в рамках тематики модуля. Лексико-грамматическое тестирование & 8\\
11 & Operating systems & СР, ПР & Практика перевода с английского языка на русский и наоборот. 
Выполнение практических заданий по всем видам речевой деятельности в рамках тематики модуля. 
Поиск информации в интернете о типах операционных систем. Написание доклада по результатам исследования. Лексико-грамматическое тестирование & 8 \\
12 & Computer storage &	СР, ПР & Практика перевода с английского языка на русский. Выполнение практических заданий по всем видам речевой деятельности в рамках тематики модуля. Поиск информации в интернете о видах инфляции. Написание доклада по результатам исследования. Лексико-грамматическое тестирование & 8\\
13 & Application programs &	СР, ПР & Практика перевода с английского языка на русский. Выполнение практических заданий по всем видам речевой деятельности в рамках тематики модуля. 
Подготовка доклада на тему «Внешняя торговля». Лексико-грамматическое тестирование & 8 \\
14 & Multimedia & СР, ПР & Практика перевода с английского языка на русский. Выполнение практических заданий по всем видам речевой деятельности в рамках тематики модуля. Лексико-грамматическое тестирование & 8\\
15 & Software engineering & СР, ПР & Практика перевода с английского языка на русский. Выполнение практических заданий по всем видам речевой деятельности в рамках тематики модуля. Лексико-грамматическое тестирование & 8\\
16 & Computer networks & СР, ПР & Практика перевода с английского языка на русский. Выполнение практических заданий по всем видам речевой деятельности в рамках тематики модуля. 
Подготовка доклада на тему «Компьютерные сети».  Лексико-грамматическое тестирование& 8 \\
17 & The internet & СР, ПР & Практика перевода с английского языка на русский и наоборот. Написание доклада о деятельности предприятий и видах собственности. Лексико-грамматическое тестирование & 8 \\
18 & Websites & СР, ПР & Практика перевода с английского языка на русский. Выполнение практических заданий по всем видам речевой деятельности в рамках тематики модуля. Написание эссе: «Веб-сайты». Лексико-грамматическое тестирование & 8 \\
19 & Communication systems & СР, ПР & Практика перевода с английского языка на русский. Выполнение практических заданий по всем видам речевой деятельности в рамках тематики модуля. Групповое исследование по тематике модуля. Лексико-грамматическое тестирование & 8 \\
20 & The future of IT & СР, ПР & Практика перевода с английского языка на русский. Выполнение практических заданий по всем видам речевой деятельности в рамках тематики модуля. Написание доклада «Будущее информационных технологий». Лексико-грамматическое тестирование & 8 \\
\end{longtable}
\end{center}

\chapter{ПЕРЕЧЕНЬ УЧЕБНО-МЕТОДИЧЕСКОГО ОБЕСПЕЧЕНИЯ ДЛЯ САМОСТОЯТЕЛЬНОЙ РАБОТЫ ОБУЧАЮЩИХСЯ ПО ДИСЦИПЛИНЕ}
\label{chapt5}
Самостоятельное изучение тем учебной дисциплины способствует закреплению знаний, умений и навыков, полученных в ходе аудиторных занятий; углублению и расширению знаний по отдельным вопросам и темам дисциплины. 
Перечень учебно-методического материала:
\begin{enumerate}
\item Беспалова Н.П. Английский язык. Грамматические трудности перевода Учеб. пособие. – М.: Дрофа, 2006. – 79 c.
\item Егошина Е.М. Английский язык [Электронный ресурс] : сборник текстов и упражнений / Е.М. Егошина. — Электрон. текстовые данные. — Йошкар-Ола: Поволжский государственный технологический университет, 2015. — 106 c. — 978-5-8158-1494-3. — Режим доступа: http://www.iprbookshop.ru/75430.html
\item Есенина Н.Е. Электронные ресурсы, распределенные в сети Интернет, на занятиях по иностранному языку неязыкового вуза: метод. указ. (Англ.яз.). – Рязань: РГРТУ, 2005. – 16 с.
\item Заеко О.В. Английский язык. Сборник текстов [Электронный ресурс] : учебно-методическое пособие / О.В. Заеко. — Электрон. текстовые данные. — М. : Московский гуманитарный университет, 2015. — 15 c. — 978-5-906822-07-9. — Режим доступа: http://www.iprbookshop.ru/50661.html
\item Клюкина Ю.В. Курс английского языка (A course of English) [Электронный ресурс] : учебное пособие для студентов всех специальностей и направлений подготовки / Ю.В. Клюкина, А.А. Шиповская. — Электрон. текстовые данные. — Тамбов: Тамбовский государственный технический университет, ЭБС АСВ, 2015. — 174 c. — 978-5-8265-1472-6. — Режим доступа: http://www.iprbookshop.ru/64105.html
\item Речевой практикум по английскому языку. Часть 1 [Электронный ресурс] : учебное пособие / А.А. Дрюченко [и др.]. — Электрон. текстовые данные. — Воронеж: Воронежский государственный университет инженерных технологий, 2016. — 272 c. — 978-5-00032-217-8. — Режим доступа: http://www.iprbookshop.ru/64413.html
\item Речевой практикум по английскому языку. Часть 2 [Электронный ресурс] : учебное пособие / А.А. Дрюченко [и др.]. — Электрон. текстовые данные. — Воронеж: Воронежский государственный университет инженерных технологий, 2016. — 156 c. — 978-5-00032-218-5. — Режим доступа: http://www.iprbookshop.ru/64414.html
\item Фомиченко А.С. English Grammar for Electrical Engineers [Электронный ресурс] : учебное пособие / А.С. Фомиченко. — Электрон. текстовые данные. — Оренбург: Оренбургский государственный университет, ЭБС АСВ, 2016. — 110 c. — 978-5-7410-1526-1. — Режим доступа: http://www.iprbookshop.ru/69882.html
\end{enumerate}

\chapter{ФОНД ОЦЕНОЧНЫХ СРЕДСТВ ДЛЯ ПРОВЕДЕНИЯ ПРОМЕЖУТОЧНОЙ АТТЕСТАЦИИ ОБУЧАЮЩИХСЯ ПО ДИСЦИПЛИНЕ}
\label{chapt6}
Фонд оценочных средств для проведения промежуточной аттестации обучающихся по дисциплине представлен в виде оценочных материалов и приведен в Приложении A.

\chapter{ПЕРЕЧЕНЬ ОСНОВНОЙ И ДОПОЛНИТЕЛЬНОЙ УЧЕБНОЙ ЛИТЕРАТУРЫ, НЕОБХОДИМОЙ ДЛЯ ОСВОЕНИЯ ДИСЦИПЛИНЫ}
\label{chapt7}
\section{Основная учебная литература (английский язык)}
\begin{enumerate}
\item «Организация курса профессионально-ориентированного иноязычного чтения в условиях информатизации образования», Е.Н. Есенина, №3676, Рязань 2005г. 
\item «Тексты для реферирования по вычислительной технике», Е.Б. Фомина, Н.Е. Есенина, З.В. Игнатова, №3466, Рязань 2003г.
\item Английский язык для инженерных факультетов = English for Engineering Faculties [Электронный ресурс] : учебник / Л.Б. Кадулина [и др.]. — Электрон. текстовые данные. — Томск: Томский государственный университет систем управления и радиоэлектроники, 2015. — 350 c. — 978-5-86889-689-7. — Режим доступа: http://www.iprbookshop.ru/72064.html
\item Башмакова И.С. Английский язык для студентов технических вузов. Modern Vehicle and Electronics: учебное пособие. – М.: Филоматис. Издательство «Омега-Л», 2010г. – 456с.
\item Дмитренко Н.А. Английский язык. Engineering sciences [Электронный ресурс] : учебное пособие / Н.А. Дмитренко, А.Г. Серебрянская. — Электрон. текстовые данные. — СПб. : Университет ИТМО, 2015. — 113 c. — 978-5-9905471-2-4. — Режим доступа: http://www.iprbookshop.ru/65782.html
\item Игнаткина И.В. English for IT students. 2 курс [Электронный ресурс] : учебное пособие по английскому языку / И.В. Игнаткина. — Электрон. текстовые данные. — Самара: Поволжский государственный университет телекоммуникаций и информатики, 2016. — 131 c. — 2227-8397. — Режим доступа: http://www.iprbookshop.ru/71818.html
\item Иностранный язык профессионального общения (английский язык) [Электронный ресурс] : учебное пособие / И.Б. Кошеварова [и др.]. — Электрон. текстовые данные. — Воронеж: Воронежский государственный университет инженерных технологий, 2018. — 140 c. — 978-5-00032-323-6. — Режим доступа: http://www.iprbookshop.ru/76428.html
\end{enumerate}
\section{Дополнительная учебная литература (английский язык)}
\begin{enumerate}
\item Беспалова Н.П. Английский язык. Грамматические трудности перевода Учеб. пособие. – М.: Дрофа, 2006. – 79 c.
\item Егошина Е.М. Английский язык [Электронный ресурс] : сборник текстов и упражнений / Е.М. Егошина. — Электрон. текстовые данные. — Йошкар-Ола: Поволжский государственный технологический университет, 2015. — 106 c. — 978-5-8158-1494-3. — Режим доступа: http://www.iprbookshop.ru/75430.html
\item Есенина Н.Е. Электронные ресурсы, распределенные в сети Интернет, на занятиях по иностранному языку неязыкового вуза: метод. указ. (Англ.яз.). – Рязань: РГРТУ, 2005. – 16 с.
\item Заеко О.В. Английский язык. Сборник текстов [Электронный ресурс] : учебно-методическое пособие / О.В. Заеко. — Электрон. текстовые данные. — М. : Московский гуманитарный университет, 2015. — 15 c. — 978-5-906822-07-9. — Режим доступа: http://www.iprbookshop.ru/50661.html
\item Клюкина Ю.В. Курс английского языка (A course of English) [Электронный ресурс] : учебное пособие для студентов всех специальностей и направлений подготовки / Ю.В. Клюкина, А.А. Шиповская. — Электрон. текстовые данные. — Тамбов: Тамбовский государственный технический университет, ЭБС АСВ, 2015. — 174 c. — 978-5-8265-1472-6. — Режим доступа: http://www.iprbookshop.ru/64105.html
\item Речевой практикум по английскому языку. Часть 1 [Электронный ресурс] : учебное пособие / А.А. Дрюченко [и др.]. — Электрон. текстовые данные. — Воронеж: Воронежский государственный университет инженерных технологий, 2016. — 272 c. — 978-5-00032-217-8. — Режим доступа: http://www.iprbookshop.ru/64413.html
\item Речевой практикум по английскому языку. Часть 2 [Электронный ресурс] : учебное пособие / А.А. Дрюченко [и др.]. — Электрон. текстовые данные. — Воронеж: Воронежский государственный университет инженерных технологий, 2016. — 156 c. — 978-5-00032-218-5. — Режим доступа: http://www.iprbookshop.ru/64414.html
\item Фомиченко А.С. English Grammar for Electrical Engineers [Электронный ресурс] : учебное пособие / А.С. Фомиченко. — Электрон. текстовые данные. — Оренбург: Оренбургский государственный университет, ЭБС АСВ, 2016. — 110 c. — 978-5-7410-1526-1. — Режим доступа: http://www.iprbookshop.ru/69882.html
\end{enumerate}
\section{Основная учебная литература (немецкий язык)}
\begin{enumerate}
\item Пасечная Л.А. Wirtschaftsdeutsch [Электронный ресурс]: учебное пособие по немецкому языку / Л.А. Пасечная, В.Е. Щербина. — Электрон. текстовые данные. — Оренбург: Оренбургский государственный университет, ЭБС АСВ, 2014. — 155 c. — 2227-8397. — Режим доступа: http://www.iprbookshop.ru/33618.html. — ЭБС «IPRbooks»
\item Ачкасова Н.Г. Немецкий язык для бакалавров [Электронный ресурс]: учебник для студентов неязыковых вузов. — М.: ЮНИТИ-ДАНА, 2014.— 312 c.— Режим доступа: http://www.iprbookshop.ru/20980.html.— ЭБС «IPRbooks»
\item Снигирева О.М. Wirtschaftsdeutsch [Электронный ресурс]: учебное пособие / О.М. Снигирева, Т.С. Талалай. — Электрон. текстовые данные. — Оренбург: Оренбургский государственный университет, ЭБС АСВ, 2015. — 117 c. — 978-5-7410-1247-5. — Режим доступа: http://www.iprbookshop.ru/52311.html. — ЭБС «IPRbooks»
\end{enumerate}
\section{Дополнительная учебная литература (немецкий язык)}
\begin{enumerate}
\item Дальке С.Г. Немецкий язык [Электронный ресурс]: учебное пособие. — Омск: Омский государственный институт сервиса, 2014. — 100 c.— Режим доступа: http://www.iprbookshop.ru/26687.html.— ЭБС «IPRbooks»
\item Немецкий язык [Электронный ресурс]: учебно-методическое пособие № 11. — СПб.: Санкт-Петербургский государственный архитектурно-строительный университет, ЭБС АСВ, 2013.— 104 c.— Режим доступа: http://www.iprbookshop.ru/19013.html.— ЭБС «IPRbooks»
\item Наседкина Г.А. Времена глагола [Электронный ресурс]: тренировочные упражнения по грамматике немецкого языка для студентов неязыковых вузов. — Челябинск: Челябинский государственный институт культуры, 2011.— 36 c.— Режим доступа: http://www.iprbookshop.ru/56397.html.— ЭБС «IPRbooks»
\item Наседкина Г.А. Страноведение Германии [Электронный ресурс]: сборник текстов по немецкому языку для студентов неязыковых вузов. — Челябинск: Челябинский государственный институт культуры, 2011.— 31 c.— Режим доступа: http://www.iprbookshop.ru/56509.html.— ЭБС «IPRbooks»
\item Иванова Л.В. Немецкий язык для профессиональной коммуникации [Электронный ресурс]: учебное пособие для самостоятельной работы студентов / Л.В. Иванова, О.М. Снигирева, Т.С. Талалай. — Электрон. текстовые данные. — Оренбург: Оренбургский государственный университет, ЭБС АСВ, 2013. — 153 c. — 2227-8397. — Режим доступа: http://www.iprbookshop.ru/30113.html. – ЭБС «IPRbooks»
\end{enumerate}
\section{Основная учебная литература (французский язык)}
\begin{enumerate}
\item Penfornis J.-L. Francais.com (Niveau Intermediaire). – Paris: CLE International, 2011. – 168 p.
\item Penfornis J.-L. Francais.com. Cahier d’exercises. – Paris: CLE International, 2011. – 102 p.
\item Голотвина Н.В. Грамматика французского языка в схемах и упражнениях [Электронный ресурс]: пособие для изучающих французский язык. — СПб.: КАРО, 2013.— 176 c.— Режим доступа: http://www.iprbookshop.ru/19381.html.— ЭБС «IPRbooks» 
\item Jean-Luc Penfornis «Français.com Français professionnel» 2-e édition CLE International,2011
\item Jean-Luc Penfornis «Français.com Cahier d’exercices» CLE International, 2011
\item Maїa Grégoire, Odile Thiévenaz «Grammaire progressive du français. Niveau intermédiaire» CLE International, 2012
\item Fanchon «Guide des Sciences et technologies industrielles», Nathan, Coll. AFNOR-NATHAN, 2011
\item Mangiante, Parpette «Le Français sur objectif universitaire, DVD-Rom inclus», PUG, 2011
\item Leroy-Miquel Claire, Lété Anne Goliot.Vocabulaire progressif du français avec 250 exercices. - CLE International,2012.
\item Jacqueline Tolas, Océane Gewirtz et Catherine Carras «Réussir ses études d’ingénieur en français», Grenoble, PUG, 2014
\end{enumerate}
\section{Дополнительная учебная литература (французский язык)}
\begin{enumerate}
\item Penfornis J.-L. Affaires.com. – Paris: CLE International, 2011. – 144 p.
\item Penfornis J.-L. Vocabulaire Progressif du Francais des Affaires avec 250 exercises. – Paris: CLE International, 2011. – 176 p.
\item Иванченко А.И. Французская грамматика в таблицах и схемах [Электронный ресурс]. – СПб.: КАРО, 2011. – 128 с. – Режим доступа: http://www.iprbookshop.ru/19471.html.— ЭБС «IPRbooks»
\item Иванченко А.И. Грамматика французского языка в упражнениях [Электронный ресурс]: 400 упражнений с ключами и комментариями. — СПб.: КАРО, 2014.— 352 c.— Режим доступа: http://www.iprbookshop.ru/19495.html.— ЭБС «IPRbooks»
\item Николаева Е.А. Французский язык [Электронный ресурс]: учебное пособие по курсу "Страноведение".— СПб.: Издательство СПбКО, 2010.— 176 c.— Режим доступа: http://www.iprbookshop.ru/11262.html.— ЭБС «IPRbooks»
\item Тетенькина Т.Ю. Французский язык [Электронный ресурс]: учебное пособие.— Электрон. текстовые данные.— Минск: Вышэйшая школа, 2010.— 287 c.— Режим доступа: http://www.iprbookshop.ru/20166.html.— ЭБС «IPRbooks»
\end{enumerate}

\chapter{ПЕРЕЧЕНЬ РЕСУРСОВ ИНФОРМАЦИОННО–ТЕЛЕКОММУНИКАЦИОННОЙ СЕТИ ИНТЕРНЕТ, НЕОБХОДИМЫХ ДЛЯ ОСВОЕНИЯ ДИСЦИПЛИНЫ}
\label{chapt8}
\section{Информационные ресурсы}
\begin{enumerate}
\item Информационная система «Единое окно доступа к образовательным ресурсам» [Электронный ресурс]. – URL: http://window.edu.ru.
\item Информационно-правовой портал ГАРАНТ.РУ [Электронный ресурс]. – URL: http://www.garant.ru. 
\item Справочная правовая система КонсультантПлюс [Электронный ресурс]. – URL: http://www.consultant.ru/online/
\item Электронно-библиотечная система IPRBookShop (http://www.iprbookshop.ru).
\item Электронно-библиотечная система "Лань" (https://e.lanbook.com).
\item Электронная библиотечная система РГРТУ (http://elib.rsreu.ru/ebs).
\item Научная электронная библиотека «CYBERLENINKA» [Электронный ресурс]: содержит электронные версии научных статей и журналов. – Режим доступа: https://cyberleninka.ru/article/c/istoriya-istoricheskie-nauki. 
\item Российская государственная библиотека [Электронный ресурс]: содержит электронные версии книг, учебников, монографий, сборников научных трудов как отечественных, так и зарубежных авторов, периодических изданий. – Режим доступа: http:// www.rbc.ru.
\item Национальная электронная библиотека [Электронный ресурс]: содержит оцифрованные версии научных монографий и статей. – Режим доступа: https://нэб.рф/https://нэб.рф/.
\item Российская Национальная Библиотека [Электронный ресурс] : официальный сайт. – Режим доступа: http://www.nlr.ru/. 
\end{enumerate}
\section{Online-словари}
\begin{enumerate}
\item Мультитран [Электронный ресурс]. – Режим доступа: www.multitran.ru
\item Словари компании ABBYY [Электронный ресурс]. – Режим доступа: http://www.lingvo.ru/
\item Longman: Dictionary of Contemporary English [Электронный ресурс]. – Режим доступа: http://www.ldoceonline.com/
\item Onelook Dictionaries [Электронный ресурс]. – Режим доступа: www.onelook.com
\item Webster’s Dictionary [Электронный ресурс]. – Режим доступа: http://www.merriam-webster.com/dictionary.htm 
\end{enumerate}
\section{Информационные ресурсы для английского языка}
\begin{enumerate}
\item http://commsystems.com/CSI/
\item http://eu.wiley.com/WileyCDA/WileyTitle/productCd-DAC.html
\item http://iecom.dee.ufcg.edu.br/~jcis/Abril%202010/index.html	
\item http://www.stmjournals.com/journals/JoCES/journal-of-telecommunication.html
\end{enumerate}
\section{Информационные ресурсы для немецкого языка}
\begin{enumerate}
\item www.goethe.de
\item www.hueber.de
\item www.langenscheidt.de
\item www.grammade.ru
\item http://de.wikipedia.org/wiki/Wikipedia:Hauptseite
\item http://wortschatz.uni-leipzig.de/
\end{enumerate}
\section{Информационные ресурсы для французского языка}
\begin{enumerate}
\item grammairefrancaise.net/
\item http://www.studyfrench.ru/
\item www.ikonet.com/fr
\item http://auberge.int.univ-lille3.fr/
\item http://www.francaisfacile.com
\item http://www.francuzskiy.fr/
\item http://www.le-francais.ru/
\item http://www.bonjour.com/
\item http://www.bbc.co.uk/languages/french/
\item http://francite.ru/
\item http://les-verbes.com/
\item http://www.bonjourdefrance.com/index/indexgram.htm
\item http://tcf.didierfle.com/
\item http://www.ladictee.fr/
\item http://www.studyfrench.ru/topics/
\item http://phonetique.free.fr/alpha.htm
\item http://www.larousse.fr/
\item http://www.arte.tv/fr
\end{enumerate}


\chapter{МЕТОДИЧЕСКИЕ УКАЗАНИЯ ДЛЯ ОБУЧАЮЩИХСЯ ПО ОСВОЕНИЮ ДИСЦИПЛИНЫ}
\label{chapt9}

\section{Рекомендации по планированию и организации времени, необходимого для изучения дисциплины}
Рекомендуется следующим образом организовать время, необходимое для изучения дисциплины:
\begin{itemize}
\item изучение конспекта лекции в тот же день, после лекции – 10-15 минут;
\item изучение конспекта лекции за день перед следующей лекцией – 10-15 минут;
\item изучение теоретического материала по учебнику и конспекту – 1 час в неделю.
\end{itemize}

\section{Описание последовательности действий студента}
При изучении дисциплины очень полезно самостоятельно изучать материал, который еще не прочитан на лекции не применялся на лабораторном занятии. Тогда лекция будет гораздо понятнее. Однако легче при изучении курса следовать изложению материала на лекции. Для понимания материала и качественного его усвоения рекомендуется такая последовательность действий:
\begin{enumerate}
\item После прослушивания лекции и окончания учебных занятий, при подготовке к занятиям следующего дня, нужно сначала просмотреть и обдумать текст лекции, прослушанной сегодня (10-15 минут).
\item При подготовке к следующей лекции, нужно просмотреть текст предыдущей лекции, подумать о том, какая может быть тема следующей лекции (10-15 минут).
\item В течение недели выбрать время (минимум 1час) для работы с литературой в библиотеке, а также в сети Интернет.
\end{enumerate}
\section{Рекомендации по работе с литературой}
Теоретический материал курса становится более понятным, когда дополнительно к прослушиванию лекции и изучению конспекта, изучаются и книги по педагогике высшей школы. Литературу по курсу рекомендуется изучать в библиотеке. Полезно использовать несколько учебников по курсу. Рекомендуется после изучения очередного параграфа ответить на несколько простых вопросов по данной теме. Кроме того, очень полезно мысленно задать себе следующие вопросы (и попробовать ответить на них): «о чем этот параграф?», «Какие новые понятия введены, каков их смысл?».
\section{Рекомендации подготовке к зачёту и экзамену}
В процессе подготовки к зачету и экзамену рекомендуется:
\begin{itemize}
\item повторно прочитать и перевести наиболее трудные тексты из учебника;
\item просмотреть материал отрецензированных тестов и контрольных работ;
\item проделать выборочно отдельные лексико-грамматические упражнения из учебника для самопроверки;
\item повторить активную лексику;
\item перевести тексты по внеаудиторному чтению; тексты для внеаудиторного чтения должны быть переведены устно. При проверке внеаудиторного чтения студент должен предъявить выписанные незнакомые слова, которыми он может пользоваться при ответе;
\item повторить устные темы.
\end{itemize}

\chapter{ПЕРЕЧЕНЬ ИНФОРМАЦИОННЫХ ТЕХНОЛОГИЙ, ИСПОЛЬЗУЕМЫХ ПРИ ОСУЩЕСТВЛЕНИИ ОБРАЗОВАТЕЛЬНОГО ПРОЦЕССА ПО ДИСЦИПЛИНЕ}
\label{chapt10}
При реализации программы бакалавриата применяются элементы электронного обучения, под которым понимается организация образовательной деятельности с применением содержащейся в базах данных и используемой при реализации образовательных программ информации и обеспечивающих ее обработку информационных технологий, технических средств, а также информационно-телекоммуникационных сетей, обеспечивающих передачу по линиям связи указанной информации, взаимодействие обучающихся и педагогических работников. При проведении занятий по дисциплине используются следующие элементы электронного обучения:
\begin{itemize}
\item удаленные информационные коммуникации между студентами и преподавателем, ведущим практические занятия, посредством электронной почты, позволяющие осуществлять оперативный контроль графика выполнения и содержания контрольных заданий, решение организационных вопросов, удаленное консультирование;
\item поиск актуальной научной, статистической и общественно-политической информации для выполнения самостоятельной работы и контрольных заданий;
\item доступ к современным профессиональным базам данных (в том числе международным реферативным базам данных научных изданий) и информационным справочным системам.
\end{itemize}

В учебном процессе применяются следующие информационные технологии:
\begin{itemize}
\item выполнение студентами заданий с использованием лицензионного или свободно распространяемого программного обеспечения, установленного на рабочих местах студента в компьютерных классах и в помещениях для самостоятельной работы, а также для выполнения самостоятельной работы в домашних условиях.
\end{itemize}

К числу информационных технологий, программ и программного обеспечения, наличие которых необходимо для успешного изучения студентами учебной дисциплины, следует отнести:
\begin{enumerate}
\item Операционная система Windows XP (Microsoft Imagine, номер подписки 700102019, бессрочно)
\item Kaspersky Endpoint Security (Коммерческая лицензия на 1000 компьютеров № 2304-180222-115814-600-1595, срок действия с 25.02.2018 по 05.03.2019)
\item Apache OpenOffice 4.1.5 (лицензия: Apache License 2.0).
\end{enumerate}

\chapter{ОПИСАНИЕ МАТЕРИАЛЬНО-ТЕХНИЧЕСКОЙ БАЗЫ, НЕОБХОДИМОЙ ДЛЯ ОСУЩЕСТВЛЕНИЯ ОБРАЗОВАТЕЛЬНОГО ПРОЦЕССА ПО ДИСЦИПЛИНЕ}
Для освоения дисциплины необходима следующая материально-техническая база.
\begin{enumerate}
\item Учебная аудитория для проведения занятий лекционного типа, занятий семинарского типа, практических занятий, в том числе выполнения учебных, курсовых и дипломных работ, групповых и индивидуальных консультаций, текущего контроля и промежуточной аттестации, имеющая следующее оборудование: 
	\begin{itemize}
		\item маркерная (меловой) доска;
		\item средства отображения презентаций (мультимедийный проектор, экран, компьютер/ноутбук, лицензионное или свободно-распространяемое программное обеспечение). 
	\end{itemize}
\item Учебная аудитория для проведения практических занятий и самостоятельной работы, имеющая следующее оборудование: 
	\begin{itemize}
		\item индивидуальная компьютерная техника с подключением к локальной вычислительной сети и сети Интернет;
		\item лингво-мультимедийная лаборатория, оборудованная средствами отображения учебных материалов на экран;
		\item среда Moodle для проведения дистанционного обучения и консультаций.
	\end{itemize}	
\item Аудитория для самостоятельной работы с возможностью подключения к сети «Интернет» и обеспечением доступа в электронную информационно-образовательную среду РГРТУ.
\end{enumerate}
