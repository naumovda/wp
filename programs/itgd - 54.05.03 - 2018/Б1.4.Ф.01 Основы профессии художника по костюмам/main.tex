\chapter{ПЕРЕЧЕНЬ ПЛАНИРУЕМЫХ РЕЗУЛЬТАТОВ ОБУЧЕНИЯ ПО ДИСЦИПЛИНЕ, СООТНЕСЕННЫХ С ПЛАНИРУЕМЫМИ РЕЗУЛЬТАТАМИ ОСВОЕНИЯ ОБРАЗОВАТЕЛЬНОЙ ПРОГРАММЫ} \label{chapt1}
Рабочая программа по дисциплине \DisciplineName~ является составной частью основной профессиональной образовательной программы по специальности \SpecialityCode~\SpecialityName, разработанной в соответствии с Федеральным государственным образовательным стандартом высшего образования по специальности \SpecialityCode~\SpecialityName~ (\QualificaionLevel), утвержденным \DocumentApprove.

Цель дисциплины \DisciplineName ~-– помочь студентам изучить основы профессии художника по костюмам в процессе работы над созданием образа персонажа.

Основные задачи освоения учебной дисциплины:
\begin{enumerate}
\item Познакомить студентов механизмом работы над образом персонажа.
\item Изучить творческий путь художников по костюмам в современном кино.
\end{enumerate}

В таблице \ref{tblCompetentions} приведены коды компетенций, содержание компетенций и перечень планируемых результатов обучения по дисциплине.

\begin{table} [ht]%
	\caption{Компетенции дисциплины}
	\label{tblCompetentions}	
	\begin{tabularx}{\textwidth}{p{.1\textwidth}p{.3\textwidth}X}
        \toprule
    	Код & Содержание \par компетенций & Перечень планируемых результатов обучения по дисциплине\\
        \midrule 
  		ПК-12 & способность критически переосмысливать накопленный опыт, изменять при необходимости профиль своей профессиональной деятельности & Знать: \par – профиль своей профессиональной деятельности; \par Уметь: \par – изменять при необходимости профиль своей профессиональной деятельности в рамках смежных творческих направлений; \par Владеть: \par – способностью с легкостью и гибкостью критически переосмысливать накопленный опыт.\\
        \bottomrule
	\end{tabularx}
\end{table}


\chapter{МЕСТО ДИСЦИПЛИНЫ В СТРУКТУРЕ ОБРАЗОВАТЕЛЬНОЙ ПРОГРАММЫ}\label{chapt2}

Дисциплина \DisciplineName~ является \DisciplineNeed~дисциплиной, относится к блоку №\BlockCode~основной профессиональной образовательной программы (ОПОП) по направлению подготовки 
\SpecialityCode~\SpecialityName~\OrganizationShortName.

Дисциплина изучается по формам обучения: \FormOfStudy, в семестрах: \SemestryList.

\chapter{ОБЪЕМ ДИСЦИПЛИНЫ И ВИДЫ УЧЕБНОЙ РАБОТЫ}
\label{chapt3}

В разделе указан объем дисциплины в зачетных единицах с указанием количества академических часов, выделенных на контактную работу обучающихся с преподавателем (по видам занятий) и на самостоятельную работу обучающихся.

Общая трудоемкость дисциплины составляет в 2 зачетных единицы (ЗЕ).

Объем дисциплины в зачетных единицах с указанием количества академических часов, выделенных на контактную работу обучающихся с преподавателем (по видам занятий) и на самостоятельную работу обучающихся приведен в таблице \ref{tblHours}.

\begin{table}
  \caption{Трудоемкость дисциплины}
  \label{tblHours}  
  \begin{tabularx}{\textwidth}{p{.1\textwidth} X p{.1\textwidth}}
  \toprule  
  № & Вид учебной работы & Часы\\
  \midrule   
  1 & Общая трудоемкость дисцилины,\par в том числе: & 72\\
  \midrule   
  1.1 & контактная работа обучающихся  (всего),\par в том числе: & 36\\  
  \midrule     
  1.1.1 & - лекции & ~--\\    
  1.1.2 & - лабораторные работы & ~--\\  
  1.1.3 & - практические занятия & 36\\    
  \midrule   
  1.2 & самостоятельная работа обучающихся (всего),\par в том числе: & 36\\    
  \midrule     
  1.2.1 & - курсовой проект & ~--\\      
  1.2.2 & - курсовая работа & ~--\\        
  1.2.3 & - контрольная работа & ~--\\          
  1.2.4 & - подготовка к экзамену, консультации & ~--\\            
  1.2.5 & - консультации в семестре & 5\\              
  1.2.6 & - иные виды самостоятельной работы & 31\\  
  \bottomrule  
  \end{tabularx}  
\end{table}

Вид промежуточной аттестации обучающихся: зачет (3 семестр).

\chapter{СОДЕРЖАНИЕ ДИСЦИПЛИНЫ}
\label{chapt4}

В разделе приведено содержание дисциплины, структурированное по темам (разделам) с указанием отведенного на них количества академических часов и видов учебных занятий

\section{Содержание дисциплины, структурированное по темам}

В структурном отношении программа дисциплины представлена следующими темами:
\begin{enumerate}
\item Художник по костюмам: Дженни Биван.
\item Художник по костюмам: Ивонн Блейк.
\item Художник по костюмам: Марк Бриджес.
\item Художник по костюмам: Шей Канлифф.
\item Художник по костюмам: Шерон Дэвис.
\item Художник по костюмам: Линди Хэмминг.
\item Художник по костюмам: Джоанна Джонстон.
\item Художник по костюмам: Майкл Каплан.
\item Художник по костюмам: Джудианна Маковски.
\item Художник по костюмам: Маурицио Миленотти.
\item Художник по костюмам: Эллен Мирожник.
\item Художник по костюмам: Эджи Джеральд Роджерс.
\item Художник по костюмам: Пенни Роуз.
\item Художник по костюмам: Джули Вайсс.
\item Художник по костюмам: Дженти Йейтс.
\item Художник по костюмам: Мэри Зофрес.
\item Художники по костюмам: наследие.
\end{enumerate}

\section{Тематический план дисциплины}
Тематический план дисциплины (таблица \ref{tbl:Plan}) включает следующие формы учебного процесса:
\begin{itemize}
\item лекции (ЛЕК); 
\item лабораторные работы (ЛАБ); 
\item практические занятия (ПЗ); 
\item самостоятельную работу (СР).
\end{itemize}

\begin{table} [ht]%
	\caption{Тематический план}%
	\label{tbl:Plan}
    %\setlength\extrarowheight{6pt} 
    %\setlength{\tymin}{1.9cm}
	\begin{tabulary}{\textwidth} {@{} >{\zz}C >{\zz}L >{\zz}C >{\zz}C >{\zz}C >{\zz}C >{\zz}C @{}}
        \toprule
    	№ & Тема & Всего, час. & ЛЕК, час.& ЛАБ, час.& ПЗ, час.& СР, час.\\
        \midrule 
        1 & Дженни Биван & 4 & 0 & 0 & 2 & 2 \\
  		2 & Ивонн Блейк & 4 & 0 & 0 & 2 & 2 \\  
  		3 & Марк Бриджес & 4 & 0 & 0 & 2 & 2 \\  
  		4 & Шей Канлифф & 4 & 0 & 0 & 2 & 2 \\  
  		5 & Шерон Дэвис & 4 & 0 & 0 & 2 & 2 \\  
  		6 & Линди Хэмминг & 4 & 0 & 0 & 2 & 2 \\  
  		7 & Джоанна Джонстон & 4 & 0 & 0 & 2 & 2 \\  
  		8 & Майкл Каплан & 4 & 0 & 0 & 2 & 2 \\  
  		9 & Джудианна Маковски & 4 & 0 & 0 & 2 & 2 \\  
		10 & Маурицио Миленотти & 4 & 0 & 0 & 2 & 2 \\  
		11 & Эллен Мирожник & 4 & 0 & 0 & 2 & 2 \\  
  		12 & Эджи Джеральд Роджерс & 4 & 0 & 0 & 2 & 2 \\  
  		13 & Пенни Роуз & 4 & 0 & 0 & 2 & 2 \\
  		14 & Джули Вайсс & 4 & 0 & 0 & 2 & 2 \\                                                  
  		15 & Дженти Йейтс & 4 & 0 & 0 & 2 & 2 \\  
 		16 & Мэри Зофрес & 4 & 0 & 0 & 2 & 2 \\  
  		17 & Художники по костюмам: наследие & 8 & 0 & 0 & 4 & 4 \\
        \midrule%%% тонкий разделитель
        ~ & Всего: & 72 & 0 & 0 & 36 & 36 \\  
        \bottomrule %%% нижняя линейка
	\end{tabulary}%
	%\end{longtable}
\end{table}

\chapter{ПЕРЕЧЕНЬ УЧЕБНО-МЕТОДИЧЕСКОГО ОБЕСПЕЧЕНИЯ ДЛЯ САМОСТОЯТЕЛЬНОЙ РАБОТЫ ОБУЧАЮЩИХСЯ ПО ДИСЦИПЛИНЕ}
\label{chapt5}
Самостоятельное изучение тем учебной дисциплины способствует закреплению знаний, умений и навыков, полученных в ходе аудиторных занятий; углублению и расширению знаний по отдельным вопросам и темам дисциплины. Самостоятельная работа как вид учебной деятельности может использоваться при подготовке к лекциям и семинарским занятиям.


Самостоятельное изучение тем учебной дисциплины способствует: закреплению знаний, умений и навыков, полученных в ходе аудиторных занятий; углублению и расширению знаний по отдельным вопросам и темам дисциплины; освоению умений прикладного и практического использования полученных знаний; освоению умений по дисциплине. Самостоятельная работа как вид учебной работы может использоваться в форме внеаудиторной самостоятельной работы обучающихся – при подготовке к лекциям, семинарам и практическим занятиям, написании рефератов, докладов, подготовке к зачету.

Основными видами самостоятельной работы по дисциплине являются: составление конспектов, рецензий, творческих эссе, доработка конспекта лекции с применением учебника, методической и дополнительной литературы; изучение и конспектирование первоисточников; подбор иллюстраций (примеров) к теоретическим положениям; подготовка сообщения, доклада, реферата на заданную тему, курсовой работы, самостоятельное изучение отдельных вопросов и тем курса.

\chapter{ФОНД ОЦЕНОЧНЫХ СРЕДСТВ ДЛЯ ПРОВЕДЕНИЯ ПРОМЕЖУТОЧНОЙ АТТЕСТАЦИИ ОБУЧАЮЩИХСЯ ПО ДИСЦИПЛИНЕ}
\label{chapt6}
Фонд оценочных средств для проведения промежуточной аттестации обучающихся по дисциплине представлен в виде оценочных материалов и приведен в Приложении A.

\chapter{ПЕРЕЧЕНЬ ОСНОВНОЙ И ДОПОЛНИТЕЛЬНОЙ УЧЕБНОЙ ЛИТЕРАТУРЫ, НЕОБХОДИМОЙ ДЛЯ ОСВОЕНИЯ ДИСЦИПЛИНЫ}
\label{chapt7}
\section{Основная учебная литература}
\begin{enumerate}
\item Цветкова Н.Н. История текстильного искусства и костюма. Древний мир [Электронный ресурс]: учебное пособие/ Цветкова Н.Н.- Электрон. текстовые данные.- СПб.: Издательство СПбКО, 2010.- 120 c.- Режим доступа: http://www.iprbookshop.ru/11268.html.- ЭБС "IPRbooks".
\end{enumerate}
\section{Дополнительная учебная литература}
Не предусмотрена.

\chapter{ПЕРЕЧЕНЬ РЕСУРСОВ ИНФОРМАЦИОННО–ТЕЛЕКОММУНИКАЦИОННОЙ СЕТИ ИНТЕРНЕТ, НЕОБХОДИМЫХ ДЛЯ ОСВОЕНИЯ ДИСЦИПЛИНЫ}
\label{chapt8}
Информационные ресурсы:
\begin{enumerate}
\item Информационная система «Единое окно доступа к образовательным ресурсам» [Электронный ресурс]. – URL: http://window.edu.ru.
\item Информационно-правовой портал ГАРАНТ.РУ [Электронный ресурс]. – URL: http://www.garant.ru. 
\item Справочная правовая система КонсультантПлюс [Электронный ресурс]. – URL: http://www.consultant.ru/online/
\item Электронно-библиотечная система IPRBookShop (http://www.iprbookshop.ru).
\item Электронно-библиотечная система "Лань" (https://e.lanbook.com).
\item Электронная библиотечная система РГРТУ (http://elib.rsreu.ru/ebs).
\item Научная электронная библиотека «CYBERLENINKA» [Электронный ресурс]: содержит электронные версии научных статей и журналов. – Режим доступа: https://cyberleninka.ru/article/c/istoriya-istoricheskie-nauki. 
\item Российская государственная библиотека [Электронный ресурс]: содержит электронные версии книг, учебников, монографий, сборников научных трудов как отечественных, так и зарубежных авторов, периодических изданий. – Режим доступа: http:// www.rbc.ru.
\item Национальная электронная библиотека [Электронный ресурс]: содержит оцифрованные версии научных монографий и статей. – Режим доступа: https://нэб.рф/https://нэб.рф/.
\item Российская Национальная Библиотека [Электронный ресурс] : официальный сайт. – Режим доступа: http://www.nlr.ru/. 
\end{enumerate}

\chapter{МЕТОДИЧЕСКИЕ УКАЗАНИЯ ДЛЯ ОБУЧАЮЩИХСЯ ПО ОСВОЕНИЮ ДИСЦИПЛИНЫ}
\label{chapt9}

\section{Рекомендации по планированию и организации времени, необходимого для изучения дисциплины}
Рекомендуется следующим образом организовать время, необходимое для изучения дисциплины:
\begin{itemize}
\item Изучение конспекта лекции в тот же день, после лекции – 10-15 минут.
\item Изучение конспекта лекции за день перед следующей лекцией – 10-15 минут.
\item Изучение теоретического материала по учебнику и конспекту – 1 час в неделю.
\end{itemize}

\section{Описание последовательности действий студента}
При изучении дисциплины очень полезно самостоятельно изучать материал, который еще не прочитан на лекции не применялся на лабораторном занятии. Тогда лекция будет гораздо понятнее. Однако легче при изучении курса следовать изложению материала на лекции. Для понимания материала и качественного его усвоения рекомендуется такая последовательность действий:
\begin{enumerate}
\item После прослушивания лекции и окончания учебных занятий, при подготовке к занятиям следующего дня, нужно сначала просмотреть и обдумать текст лекции, прослушанной сегодня (10-15 минут).
\item При подготовке к следующей лекции, нужно просмотреть текст предыдущей лекции, подумать о том, какая может быть тема следующей лекции (10-15 минут).
\item В течение недели выбрать время (минимум 1час) для работы с литературой в библиотеке.
\end{enumerate}
\section{Рекомендации по работе с литературой}
Теоретический материал курса становится более понятным, когда дополнительно к прослушиванию лекции и изучению конспекта, изучаются и книги по педагогике высшей школы. Литературу по курсу рекомендуется изучать в библиотеке. Полезно использовать несколько учебников по курсу. Рекомендуется после изучения очередного параграфа ответить на несколько простых вопросов по данной теме. Кроме того, очень полезно мысленно задать себе следующие вопросы (и попробовать ответить на них): «о чем этот параграф?», «Какие новые понятия введены, каков их смысл?».

\chapter{ПЕРЕЧЕНЬ ИНФОРМАЦИОННЫХ ТЕХНОЛОГИЙ, ИСПОЛЬЗУЕМЫХ ПРИ ОСУЩЕСТВЛЕНИИ ОБРАЗОВАТЕЛЬНОГО ПРОЦЕССА ПО ДИСЦИПЛИНЕ}
\label{chapt10}

К числу информационных технологий, программ и программного обеспечения, наличие которых необходимо для успешного изучения студентами учебной дисциплины, следует отнести:
\begin{enumerate}
\item Операционная система Windows XP (Microsoft Imagine, номер подписки 700102019, бессрочно)
\item Kaspersky Endpoint Security (Коммерческая лицензия на 1000 компьютеров № 2304-180222-115814-600-1595, срок действия с 25.02.2018 по 05.03.2019)
\item Apache OpenOffice 4.1.5 (лицензия: Apache License 2.0).
\end{enumerate}

\chapter{ОПИСАНИЕ МАТЕРИАЛЬНО-ТЕХНИЧЕСКОЙ БАЗЫ, НЕОБХОДИМОЙ ДЛЯ ОСУЩЕСТВЛЕНИЯ ОБРАЗОВАТЕЛЬНОГО ПРОЦЕССА ПО ДИСЦИПЛИНЕ}
Для освоения дисциплины необходимы:
\begin{enumerate}
\item Учебная аудитория для проведения занятий лекционного типа, занятий семинарского типа, практических занятий, в том числе выполнения учебных, курсовых и дипломных работ, групповых и индивидуальных консультаций, текущего контроля и промежуточной аттестации, имеющая следующее оборудование: 
	\begin{itemize}
		\item специализированная мебель, 
		\item рабочее место для преподавателя, оснащенное компьютером и мультимедийным проектором;	
	\end{itemize}
\item Аудитория для самостоятельной работы с возможностью подключения к сети «Интернет» и обеспечением доступа в электронную информационно-образовательную среду РГРТУ.
\end{enumerate}
