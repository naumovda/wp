\appendix
%Чтобы приложения русскими буквами нумеровались
\renewcommand\thechapter{\Asbuk{chapter}}
%%%Оформление заголовков приложений ближе к ГОСТ:
%\setlength{\midchapskip}{20pt}
%\renewcommand*{\afterchapternum}{\par\nobreak\vskip \midchapskip}
   % Предварительные настройки для правильного подключения Приложений
\chapter{ФОНД ОЦЕНОЧНЫХ СРЕДСТВ} \label{AppendixA}

\section{Общие положения} \label{sectionA1}
Оценочные средства (ОС) – это совокупность учебно-методических материалов (контрольных заданий, описаний форм и процедур проверки), предназначенных для оценки качества освоения обучающимися данной дисциплины как части ОПОП. Оценочные средства предназначены для контроля и оценки образовательных достижений обучающихся, освоивших программу учебной дисциплины. 

Цель фонда оценочных средств (ФОС) – предоставить объективный механизм оценивания соответствия знаний, умений и владений, приобретенных обучающимся в процессе изучения дисциплины, целям и требованиям ОПОП в ходе проведения текущего контроля и промежуточной аттестации.

Основная задача ФОС – обеспечить оценку уровня сформированности общекультурных, общепрофессиональных, профессиональных и профессионально-специализированных компетенций.

\section{Перечень компетенций с указанием этапов их формирования в процессе освоения образовательной программы}

В таблице \ref{tblCompetentionsA} представлен перечень компетенций, формируемых дисциплиной.

\begin{table} [ht]%
	\caption{Компетенции дисциплины}
	\label{tblCompetentionsA}	
	\begin{tabularx}{\textwidth}{p{.1\textwidth}X}
        \toprule
    	\textbf{Код} & \textbf{Содержание} \par \textbf{компетенций}\\
        \midrule 
  		ПК-12 & способность критически переосмысливать накопленный опыт, изменять при необходимости профиль своей профессиональной деятельности\\
        \bottomrule
	\end{tabularx}
\end{table}

В таблице \ref{tblCompetention1} представлены этапы формирования компетенций в процессе освоения основной профессиональной образовательной программы.

\begin{table} [ht]%
	\caption{Этапы формирования компетенции ПК-12}
	\label{tblCompetention1}	
	\begin{tabularx}{\textwidth}{p{.1\textwidth}p{.3\textwidth}|X|X|X|X|X|X|X|X|X|X|X|X|X}
        \toprule
        \multicolumn{2}{c}{\textbf{Дисциплина}}&\multicolumn{13}{c}{\textbf{Семестры}}\\
        \midrule        
    	Код & Наименование &1&2&3&4&5&6&7&8&9&10&11&12&13\\
        \midrule 
        Б1.4.Ф.01 & Основы профессии художника по костюмам & & &+& & & & & & & & & & \\        
        \midrule         
  		Б2.Б.04 & Научно-производственная практика & & & & & & & &+& & & & & \\        
        \midrule   		
  		Б1.1.В.01 & Психология и педагогика & & & & & & & & &+& & & & \\ 
        \midrule   		
  		Б1.3.Б.07 & Основы психологии творческого процесса & & & & & & & & & &+& & & \\
        \midrule   		
  		Б2.Б.07	& Преддипломная практика & & & & & & & & & & & &+&+\\
        \midrule   		
  		Б3.Б.01	& Подготовка к процедуре защиты и процедура защиты выпускной квалификационной работы & & & & & & & & & & & & &+\\
        \bottomrule
	\end{tabularx}
\end{table}

В таблице \ref{tblPartsA} приведен перечень этапов обучения дисциплины. 

\begin{table} [ht]%
	\caption{Этапы освоения дисциплины}
	\label{tblPartsA}	
	\begin{tabularx}{\textwidth}{p{.1\textwidth}X}
        \toprule
    	\textbf{№ п/п} & \textbf{Этап обучения (разделы дисциплины)}\\
        \midrule 
		1 & Художники-постановщики: интервью \\
		2 & Основы профессии художника-постановщика \\
        \bottomrule
	\end{tabularx}
\end{table}

В таблице \ref{tblPartSkillsA} представлены этапы формирования компетенций и их частей в процессе освоения дисциплины. 

\begin{table} [ht]%
	\caption{Этапы формирования компетенций}
	\label{tblPartSkillsA}	
	\begin{tabularx}{\textwidth}{p{.05\textwidth}p{.15\textwidth}p{.1\textwidth}Xp{.1\textwidth}p{.1\textwidth}}	
	\toprule
	\multirow{2}{.05\textwidth}{№} & \multirow{2}{.15\textwidth}{Код компетенции} & \multicolumn{2}{c}{Планируемые результаты обучения} & \multicolumn{2}{c} {Этапы обучения} \\
	~&~& Код & Результат обучения & 1 & 2\\	
	\midrule
	1 & ПК-12 & З-1 & Знать профиль своей профессиональной деятельности & + & \\ 
    \midrule	
	2 & ПК-12 & У-1 & Уметь изменять при необходимости профиль своей профессиональной деятельности в рамках смежных творческих направлений & & +\\
    \midrule	
	3 & ПК-12 & В-1 & Владеть способностью критически переосмысливать накопленный опыт & + & + \\	
	\bottomrule
  \end{tabularx} 	
\end{table}

Перечень видов оценочных средств, используемых в ФОС дисциплины, представлен в таблице \ref{tblTaskKinds}.

\begin{table} [ht]%
	\caption{Перечень видов оценочных средств, используемых в процессе освоения 
дисциплины}
	\label{tblTaskKinds}	
	\begin{tabularx}{\textwidth}{p{.1\textwidth}p{.2\textwidth}X p{.2\textwidth}}
        \toprule
    	\textbf{№}&\textbf{Наименование вида оценочного средства)}& \textbf{Характеристика
оценочного средства)}&\textbf{Представление оценочного средства в ФОС}\\
        \midrule 
		1 &	Устный опрос & Средство контроля, организованное как специальная беседа преподавателя с обучающимся на темы, связанные с изучаемой дисциплиной, и рассчитанное на выяснение объема знаний обучающегося по определенному разделу, теме, проблеме и т.п	& Контрольные вопросы по темам/разделам дисциплины\\
        \midrule		
		2 & Практическое задание & Средство оценки умения применять полученные теоретические знания в практической ситуации. Задача должна быть направлена на оценивание тех компетенций, которые подлежат освоению в данной дисциплине, должна содержать четкую инструкцию по выполнению или алгоритм действий & Комплект задач и заданий\\
        \bottomrule
	\end{tabularx}
\end{table}

В паспорте фонда оценочных материалов \ref{tblPassoprtFOS} приведено соответствие между контролируемыми компетенциями и оценочными средствами контроля компетенции.

%\begin{landscape}

\clearpage

\chapter{Паспорт фонда оценочных средств}
\section{Оценочные средства компетенции ПК-12}
\textbf{ПК-12}~-- способность критически переосмысливать накопленный опыт, изменять при необходимости профиль своей профессиональной деятельности.

\begin{longtable}[c]{p{.1\textwidth}|p{.2\textwidth}|p{.2\textwidth}|p{.2\textwidth}|p{.2\textwidth}}
	\caption{Паспорт фонда оценочных средств ПК-12}
	\label{tblPassoprtFOS}\\
	\toprule
    Навык & Пороговый & Базовый & Повышенный & Оценочное средство    \\ 
    \midrule 
    \endhead
	знает & профиль своей профессиональной деятельности & профиль своей профессиональной деятельности& профиль своей профессиональной деятельности&Устный опрос\\
	\midrule
	умеет &	менять профиль своей профессиональной деятельности в рамках специальности &	изменять при необходимости профиль своей профессиональной деятельности в рамках смежных творческих направлений & изменять при необходимости профиль своей профессиональной деятельности в рамках смежных творческих направлений&Практические задания\\
	\midrule
	владеет & на базовом уровне способностью критически переосмысливать накопленный опыт & способностью критически переосмысливать накопленный опыт & способностью с легкостью и гибкостью критически переосмысливать накопленный опыт&Практические задания\\			
    \bottomrule
\end{longtable}

%\end{landscape}

